\documentclass{article}
\usepackage[utf8]{inputenc}


\title{Hanabi}
\author{Charles Dunn}
\date{July 2018}

\usepackage{natbib}
\usepackage{graphicx}
\usepackage{amsmath}
\usepackage{dsfont}
\usepackage{bm}
\usepackage{amssymb}
\usepackage{bbm}

\setcounter{section}{-1}

\DeclareMathOperator*{\argmax}{arg\,max}
\DeclareMathOperator*{\argmin}{arg\,min}
\setcounter{MaxMatrixCols}{13}

\newcommand{\Conv}{\mathop{\scalebox{1.5}{\raisebox{-0.2ex}{$\ast$}}}}%


\begin{document}

\maketitle

\section{Introduction}

Hanabi is a multiplayer cooperative card game in which players hold their cards facing out. It is the opposite of traditional card games in that you can see everyone else's cards, but not your own. The goal is to play, discard, and give clues to other players such that cards are collectively played in a valid order on the table. There are some surprisingly complicated mathematically ideas involved in the game, perhaps not to play it casually, but certainly to even approach optimality.

We set out to solve two questions about probability and combinatorics that became important while coding strategies for a computer to play Hanabi. 

First, how many combinations of hands are possible? Since the validity of a clue in the game depends on the set of cards in your hand, not on their order, we consider specifically the number of combinations, not the number of permutations. The number of cards in a player's hand varies depending on the number of players and it is interesting to know the number of combinations of cards possible across all players' hands, so we would like an analytical formula for the number of combinations of any number of cards dealt from a Hanabi deck. Since the starting Hanabi deck itself changes across different game variants and the formula will be useful when the deck has been somewhat depleted during gameplay, we would like the formula to apply for any deck of cards. Note that the standard and variant Hanabi decks have repeated elements (e.g. there are three identical 1s of each color suit). We would like the analytical solution to lead to a fast, implementable algorithm.

Second, for an unknown card what is the conditional probability distribution given partial information about the card itself and partial information on other unknown cards from the same deck? In Hanabi, you give clues such as "This card is a 1" or "These two cards are Green". You therefore know positive information about some of the cards in your hand, as well as negative information about cards that were not clued (e.g. The other cards in my hand are \emph{not} 1s"). This is of course in addition to the knowledge during the game about the cards that are visible on the table, in the discard pile, and in other players' hands. It is therefore extremely useful to be able to calculate exactly the odds of any unknown card being any card type from the deck. This is useful whether the card in question is in your hand, in the deck, or even in someone else's hand if you pretended you couldn't see its value. Like the first question, we would like an analytical solution as well as a useful algorithm.

\setcounter{subsection}{-1}

\subsection{Notation}

Most of the notation that follows is standard and clear, but for reference here are some clarifications. Non-scalar values will be bold letters. Matrices will be bold, upper-case letters (e.g. $\bm{X}$). Vectors will be bold, lower-case letters (e.g. $\bm{x}$). Both matrices and vectors are 1-indexed by bracketed, non-bold scalars (e.g. $X[m, n]$).

$\mathbb{Z}_{+}$ is the set of positive integers. $\mathbb{Z}_{*}$ is the set of non-negative integers.

We use the indicator function denoted by $\mathbbm{1}_{(\cdot)}$.

\begin{equation}
    \mathbbm{1}_{A}(x) = \begin{cases} 
      1 & x\in A \\
      0 & \text{else}
   \end{cases}
\end{equation} 

We use the similar Dirac measure as a vector denoted by $\bm{\delta_{(\cdot)}}$.

\begin{equation}
    \delta_m[n] = \begin{cases} 
      1 & n=m \\
      0 & \text{else}
   \end{cases}
\end{equation} 

The discrete Heaviside or unit step function denoted by $H_{(\cdot)}$ will be useful, as will a modified rectangular function $\Pi_{(\cdot)}$.

\begin{equation}
    H_m[n] = \begin{cases} 
      1 & n\geq m \\
      0 & \text{else}
   \end{cases} = \sum_{i = -\infty}^{\infty} \delta_m[i]
\end{equation} 

\begin{equation}
    \Pi_m[n] = \begin{cases} 
      1 & 0\leq n \leq m \\
      0 & \text{else}
   \end{cases} = H_0[n] - H_m[n - 1]
\end{equation} 


\subsection{Deck Notation}

We need to represent the cards in a Hanabi deck before we continue. Let there be $N$ cards in a Hanabi deck. Each card has a color type (i.e. a suit) from a set of colors $A_\cap$ and a number value from a set of numbers $A_\#$.\footnote{We are using $\cap$ to represent the colors, or suits, in the Hanabi deck because it looks like a rainbow.} Let $N_\cap = \lvert A_\cap \rvert$ be the number of color types and $N_\# = \lvert A_\# \rvert$ be the number of number types.

Let the deck's card type matrix be $\bm{X}\in \mathbb{Z}_{*}^{N_\cap \times N_\#}$ where $X[m, n]$ is the number of cards with color type index $m\in [1, N_\cap]$ and number value index $n\in [1, N_\#]$. We can calculate some useful values from the card type matrix.

\begin{equation}
    N = \sum_{m = 1}^{N_\cap} \sum_{n = 1}^{N_\#}X[m, n]
\end{equation} 

Let $U$ be the number of unique card types in a deck. $U=N$ if and only if each card is unique.

\begin{equation}
    U = \sum_{m = 1}^{N_\cap} \sum_{n = 1}^{N_\#} \min(X[m, n]), 1) = \sum_{m = 1}^{N_\cap} \sum_{n = 1}^{N_\#} \mathbbm{1}_{\mathbb{Z}_{+}}(X[m, n])
\end{equation} 

Let $R$ represent the highest frequency of any card type in the deck. $R=1$ for a deck with unique cards. $R=N$ if there is only one card type in the entire deck.

\begin{equation}
    R = \max (\bm{X} )
\end{equation} 

 We define a characteristic frequency vector $\bm{\rho}\in \mathbb{Z}_{*}^{R}$ whose $r$th element is the number of distinct card types with $r$ cards in the deck. This is just a histogram of the card type matrix.
 
\begin{equation}
    \rho[r] = \sum_{m = 1}^{N_\cap} \sum_{n = 1}^{N_\#} \mathbbm{1}_{r}(X[m, n])
\end{equation} 
 
Note that we can still derive all relevant metrics from this frequency vector. In fact, we only need the characteristic frequency vector, not the full card type matrix, for some of the coming combinatorics.

\begin{equation}
    R = \lvert\bm{\rho}\rvert
\end{equation}

A more robust equation would be $R = \argmax_r{ \left ( r\mathbbm{1}_{>0}(\rho[r])\right )}$, but if we always truncate $\bm{\rho}$ to its last non-zero element, than the simpler equation above suffices.

\begin{equation}
    N = \sum_{r=1}^{R} r \rho[r]
\end{equation}

\begin{equation}
    U = \sum_{r=1}^{R} \rho[r]
\end{equation}

Note that we could represent a standard fifty-two card deck with this notation, even if our notation is a bit heavy for a deck with unique cards. 

\begin{equation}
A_\cap = \{\clubsuit, \diamondsuit, \heartsuit, \spadesuit\}, 
A_{\#} = \{2, 3, 4, ..., \text{Q}, \text{K}, \text{A}\}
\end{equation}

\begin{equation}
    \bm{X} = \begin{bmatrix} 1 & 1 & 1 & 1 & 1& 1 & 1 & 1 & 1& 1 &1 & 1 & 1\\1 & 1 & 1 & 1 & 1& 1 & 1 & 1 & 1& 1 & 1 & 1 & 1\\1 & 1 & 1 & 1 & 1& 1 & 1 & 1 & 1& 1 & 1 & 1 & 1\\1 & 1 & 1 & 1 & 1& 1 & 1 & 1 & 1& 1 & 1 & 1 & 1
\end{bmatrix}
\end{equation}

\begin{equation}
    \bm{\rho} = [52]
\end{equation}

\begin{equation}
R = 1, N = 52, U = 52
\end{equation}

More usefully, we can now specify the standard 50 card Hanabi deck using the subscript $_0$ to denote its associated representation.

\begin{equation}
A_{\cap_0} = \{\text{Red}, \text{Yellow}, \text{Green}, \text{Blue}, \text{White}\}, 
A_{\#_0} = \{1, 2, 3, 4, 5\}
\end{equation}

\begin{equation}
    \bm{X_0} = \begin{bmatrix} 3 & 2 & 2 & 2 & 1\\3 & 2 & 2 & 2 & 1\\3 & 2 & 2 & 2 & 1\\3 & 2 & 2 & 2 & 1\\3 & 2 & 2 & 2 & 1
\end{bmatrix}
\end{equation}

\begin{equation}
    \bm{\rho_0} = [5, 15, 5]^T
\end{equation}

\begin{equation}
R_0 = 3, N_0 = 50, U_0 = 25
\end{equation}

\section{Number of Combinations of $K$ Cards}

As stated in the introduction, we would like to know how many combinations of hands of any size are possible. Note that we are not yet concerned with order (permutations), but just with sets of cards (combinations). The smaller this number is, the less information is needed to communicate through Hanabi actions exactly which cards and in what quantity are in a player's hand. Fundamentally, we are solving a more general question that is applicable in not only Hanabi or standard card games, but also in Scrabble or Candyland where there are letter/card types of different frequencies.

Specifically, we are looking for an analytical solution for $C(\bm{\rho}, K)$, which we define to be the number of distinct combinations of $K$ items drawn without replacement from a set with characteristic frequency $\bm{\rho}$. Ideally, we solve for it in a form that allows for efficient computation.

\subsection{Monotype Decks}

As a first step, let's look at the simple case of having any number of copies of just one card type in our deck. Let $\bm{\rho_{\delta}}$ be the characteristic frequency of such a monotype deck, with $U_{\delta} = 1$ and $N_{\delta} = R_{\delta}$.\footnote{We use the $\delta$ notation because the characteristic frequency is a delta function.}

\begin{equation}
    \rho_{\delta}[r] = \delta_{R_{\delta}}[r] = \begin{cases} 
      1 & r=R_{\delta} \\
      0 & \text{else}
   \end{cases}
\end{equation}

An equation for $C(\bm{\rho_{\delta}}, K)$, the number of combinations of $K$ cards from a monotype deck with characteristic function $\bm{\rho_{\delta}}$, is fairly simple. If there are enough cards to deal $K$ cards from the deck, then the one and only possible hand is composed of copies of the same card. If there are not enough cards to deal $K$ cards, then there are no ways to deal a hand from the deck.

\begin{equation}
    C(\bm{\rho_{\delta}}, K) = 
    \Pi_{R_{\delta}}[K] = \begin{cases} 
      1 & 0\leq K \leq R_{\delta} \\
      0 & \text{else}
   \end{cases}
\end{equation}

There are not many interesting uses of monotype decks in games, exactly because there is no randomness. As a toy example, let's say you are dealing out the 4 \$100s to each of 6 players from the set of 20 \$100 bills in Monopoly. How many combinations of bills are there in total?

\begin{equation}
    C(\bm{\delta_{20}}, 24) = \Pi_{20}[24] = 0
\end{equation}

Uhoh, looks like you need to vote someone out of the game, get more monopoly \$100s, or start using some \$500s. That probably seemed like a trivial example but don't worry, the next section is more interesting, and we will actually use the above monotype deck result in our final solution for a general equation for $C(\bm{\rho}, K)$.

\subsection{No-Repetition Decks}

Now, let's look at the simple combinatorics if we had only distinct cards in our deck. Let $\bm{\rho_{.}}$ be any such deck. The binomial coefficient is exactly what we need in this situation to solve for $C(\bm{\rho_.}, K)$. The binomial coefficient or ``n choose k'' is the number of ways to choose $k$ items from $n$ unique options without replacement.

\begin{equation}
    \binom{n}{k} = \frac{n!}{k! (n - k)!}
\end{equation}

If we want to know the number of sets of $K$ cards from a deck of $N_.$ unique cards, we use the binomial coefficient formula.

\begin{equation}
    C(\bm{\rho_{.}}, K) = \binom{N_{.}}{K} = \frac{N_{.}!}{K! (N_{.}-K)!}
\end{equation}

For a standard deck of cards, we can use this result to already solve for the number of 5-card poker hands.

\begin{equation}
    C([52]^T, 5) = \binom{52}{5} = 2,598,960
\end{equation}

\subsection{Uniform-Repetition Decks}

We can get an analytical solution of $C(\bm{\rho}, K)$ for a more general deck type. Let $\bm{\rho_{\bot}}$ be any deck with card types that have the same frequency.\footnote{We are using the $\bot$ notation because it portrays the impulse function that is the characteristic frequency of a uniform-repetition deck.} We will call these decks uniform-repetition decks since all card types are repeated the same amount. Let $\bm{\rho_\bot}$ be the characteristic frequency of a deck with $U_\bot$ card types where all cards have the same frequency $R_\bot$.

\begin{equation}
    \bm{\rho_{\bot}} = U_{\bot} \bm{\delta_{R_{\bot}}}
\end{equation}

That is, $\bm{\rho_{\bot}}$ has just one non-zero element $U_\bot$ at index $R_{\bot}$.

\begin{equation}
    \rho_{\bot}[r] = U_{\bot} \delta_{R_{\bot}}[r] = \begin{cases} U_{\bot} & r=R_{\bot} \\ 0 & \text{else}
    \end{cases}
\end{equation}

Our goal is to count the number of combinations of $K$ cards from the full deck of $N_{\bot}$ cards. In the special case that $K\leq R_\bot$, we are never restricted by running out of any card type when dealing a hand; it is possible to have a hand entirely composed of a single card type. In this special case, the number of combinations is simply the number of combinations of card types with replacement.

The number of ways to select $k$ items from a set of $n$ items with replacement is described by the following formula in the ``$n$ multichoose $k$'' notation \cite{benjamin_quinn_2003}.

\begin{equation}
    \left (\binom{n}{k}\right) = \binom{n + k - 1}{k} = \frac{(n + k - 1)!}{(n - 1)!k!}
\end{equation}

We therefore have an analytical solution for the number of possible hands when all card types in the deck have the same frequency which is at least the number of cards being selected.

\begin{equation}
    C(\bm{\rho_{\bot}}, K) = \left (\binom{U_{\bot}}{K}\right) \text{ if } K \leq R_{\bot} = \frac{(U_{\bot} + K - 1)!}{(U_{\bot} - 1)!K!} \text{ if } K \leq R_{\bot}
\end{equation}

To avoid the constraint that $K\leq R_{\bot}$, we reformulate the problem in terms of compositions. Selecting $K$ items from $U_{\bot}$ options with replacement up to $R_\bot$ times is the same as trying to place $K$ items in exactly $U_{\bot}$ bins such that each bin has a count in the range $[0, R_{\bot}]$. Each count in a bin corresponds to a card of that type being selected.

Restricted compositions are usually discussed where each bin has a count in the range $[1, R]$. Luckily, the number of compositions of $K$ items into exactly $U$ bins where each bin has a non-negative item count is the same as the number of compositions of $K + U$ items into exactly $U$ bins where each bin has a count in the range $[1, R + 1]$. Let the number of compositions of $k$ items from a set of $n$ items each selected at most $w$ times be represented by $F(n, k, w)$ \cite{abramson}.

\begin{equation}
    F(n, k, w) = \sum_{j = 0}^k(-1)^j \binom{k}{j}\binom{n - jw - 1}{k - 1}
\end{equation}

We now have an analytical solution for the number of compositions of $K$ cards from a deck with uniform frequency $\bm{\rho_{\bot}}$ regardless of the number of cards being selected.

\begin{equation}
    C(\bm{\rho_\bot}, K) = F(K + U_\bot, U_\bot, R_\bot + 1)
\end{equation}

\begin{equation}
    C(\bm{\rho_\bot}, K) = \sum_{u = 0}^{U_\bot}(-1)^u \binom{U_\bot}{u}\binom{K + U_\bot - u(R_\bot + 1) - 1}{U_\bot - 1}
\end{equation}

Note that no-repetition decks are a subset of uniform-repetition decks, so this formula will work for any deck with only unique cards as well.

If we were playing a card game with the standard playing card deck, but suits did not matter, then each card type would have exactly 4 copies in the deck. We can now count the combinations of 5 cards from this uniform-repetition deck.

\begin{equation}
    C([0, 0, 0, 13]^T, 5) = \sum_{u = 0}^{13}(-1)^u \binom{13}{u}\binom{5 + 13 - u(4 + 1) - 1}{13 - 1}
\end{equation}

\begin{equation}
    C([0, 0, 0, 13]^T, 5) = 6,175
\end{equation}

\subsection{Variable-Repetition Decks}

Let $\bm{\rho}$ be the characteristic frequency of any deck. Just like with uniform-repetition decks, variable-repetition decks have a simple equation if the number of objects being selected is fewer than the number of copies of the rarest card type, since it is effectively selection with infinite replacement.

\begin{equation}
    C(\bm{\rho}, K) = \left (\binom{U}{K}\right) \text{ if } K \leq r_{\min}
\end{equation}

For an unconstrained solution, we can again turn to restricted compositions to calculate $C(\bm{\rho}, K)$. Selecting $K$ items from $U$ options with replacement up to $r_u$ times for option $u \in [1, U]$ is the same as trying to place $U + K$ items in exactly $U$ bins such that bin $u$ has a count in the range $[1, r_u + 1]$. Each count in a bin corresponds to a card of that type being selected. Let the number of compositions of $k$ items from a set of $n$ items each selected at most $w[i]$ times be represented by $G(n, k, \bm{w})$ \cite{abramson}.

\begin{equation}
    G(n, k, \bm{w}) = \binom{n - 1}{k - 1} - \sum_{j = 1}^k (-1)^j{\sum_{}^{}} ^*\binom{n - 1 - w[i_1] - w[i_2] - ... - w[i_j]}{k - 1}
\end{equation}

Where ${\sum_{}^{}} ^*$ is the sum over all combinations of $j$ indices $i_1<i_2<...<i_j$ such that $j \leq k$. This equation is hilariously complicated and extremely not closed form. I did implement it and it produces the correct results, but generating ordered combinations of elements is unfeasible even for small $k$. Nonetheless, this is our first analytical solution to $C(\bm{\rho}, K)$ that produces the correct result without placing constraints on the structure of our characteristic function. Let $\bm{x}$ be a vectorization of the non-zero elements of the card type matrix $\bm{X}$, and note that this vector can be derived from $\bm{\rho}$ through digitization (i.e. reverse histrogamming).

\begin{equation}
    C(\bm{\rho}, K) = G(K + U, U, \bm{x} + 1)
\end{equation}

\begin{equation}
    C(\bm{\rho}, K) = \binom{K + U - 1}{U - 1} - \sum_{j = 1}^U (-1)^j{\sum_{}^{}} ^*\binom{K + U - 1 - x[i_1] - x[i_2] - ... - x[i_j]}{U - 1}
\end{equation}

Again, where ${\sum_{}^{}} ^*$ is the sum over all combinations of $j$ indices $i_1<i_2<...<i_j$ such that $j \leq U$.

It is truly an abomination, but we can calculate a useful number of combinations. Let's say we are playing Pit, a game with 9 cards each of 7 different commodities, plus two special cards, the Bear and the Bull. You are dealt 9 cards at the start of the game. How many possible starting hands are there?

\begin{equation}
    C([2, 0, 0, 0, 0, 0, 0, 0, 7]^T, 9) = 12,727
\end{equation}

We have our solution, but we initially set out to find the number of combinations of hands in standard 4-player Hanabi. Unfortunately, the equation above is so slow when implemented, that it hasn't finished calculating this result.

\begin{equation}
    C(\bm{\rho_0}, 4) = ...?...
\end{equation}

The literature has let us down a bit in this case, but we can actually solve this problem through simpler math.

\subsection{Efficient Solution}

Let $\bm{\rho}$ be the characteristic function for any deck, and let $\bm{\rho_1}$ and $\bm{\rho_2}$ be disjoint subsets of this deck such that $\bm{\rho} = \bm{\rho_1} + \bm{\rho_2}$ (i.e. all cards from the original deck of the same card type are all either in subdeck 1 or subdeck 2). $C(\bm{\rho}, K)$, $C(\bm{\rho_1}, K_1)$ and $C(\bm{\rho_2}, K_2)$ are the number of combinations of $K$, $K_1$, and $K_2$ cards from each of these decks, respectively. Since our full deck is composed entirely of cards from either subdeck, $K = K_1 + K_2$. We are interested in solving for $C(\bm{\rho}, K)$ in terms of $C(\bm{\rho_1}, K_1)$ and $C(\bm{\rho_2}, K_2)$. 

Let $C(\bm{\rho}, K; K_1)$ be the combinations of $K$ cards from the full deck given that there are exactly $K_1$ cards from the first subdeck. There are $C(\bm{\rho_1}, K_1)$ combinations of $K_1$ cards from the first subdeck, and for each of these arrangements, there are exactly the same $C(\bm{\rho_2}, K - K_1)$ arrangements of $K - K_1$ cards from the second deck. It is important that the two subdecks contain entirely distinct card types, or this would not be true.

\begin{equation}
    C(\bm{\rho}, K; K_1) = C(\bm{\rho_1}, K_1 ; K_1) C(\bm{\rho_2}, K_2; K_1)= C(\bm{\rho_1}, K_1) C(\bm{\rho_2}, K - K_1)
\end{equation}

Of course, for any $K$, $K_1$ can be anything in the range $[0, K]$. These count distinct arrangements, since no hand made of $K_a$ cards from a subdeck can ever be the same as any hand made of $K_b$ cards from the same subdeck if $K_a \ne K_b$. Again, it is important that the two subdecks contain entirely distinct card types, or this would not be true. Therefore, we just need to sum all combinations with $K_1$ cards from the first subdeck and $K - K_1$ cards from the second subdeck over all values of $K_1$.

\begin{equation}
    C(\bm{\rho}, K) = \sum_{k_1=0}^K C(\bm{\rho}, K; k_1) = \sum_{k_1=0}^K C(\bm{\rho_1}, k_1) C(\bm{\rho_2}, K - k_1)
\end{equation}

This is exactly the convolution of the two combination-counting functions, where below is the definition of the discrete convolution.

\begin{equation}
     (f * g)[x] \equiv \sum_{\gamma=-\infty}^{\infty} f[\gamma]g[x - \gamma]
\end{equation}

\begin{equation}
    C(\bm{\rho}, K) = (C(\bm{\rho_1}, \cdot) * C(\bm{\rho_2}, \cdot))[K]
\end{equation}

Since any deck with more than one card type can be recursively divided into two distinct subdecks with no cards of the same type, we can repeatedly apply this method. Let $\bm{\rho_{s}}$ be the characteristic frequency of any set of $S$ disjoint subdecks of $\bm{\rho}$.

\begin{equation}
    \bm{\rho} = \sum_{s = 1}^{S} \bm{\rho_{s}}
\end{equation}

We can then accumulate the number of combinations one subdeck at a time.

\begin{equation}
    C(\bm{\rho}, K) = (C(\bm{\rho_{1}}, \cdot) * (C(\bm{\rho_{2}},\cdot) * ... * (C(\bm{\rho_{S}}, \cdot))...))[K]
\end{equation}

Convolution is linear, and therefore associative.

\begin{equation}
    C(\bm{\rho}, K) = (C(\bm{\rho_{1}}, \cdot) * C(\bm{\rho_{2}},\cdot) * ... * C(\bm{\rho_{S}}, \cdot)) [K]
\end{equation}

And we can simplify the notation

\begin{equation}
    C(\bm{\rho}, K) =  \left (\Conv_{s = 1}^S C(\bm{\rho_{s}}, \cdot)\right )[K]
\end{equation}

This is much more elegant than our previous solution for decks with arbitrary structure. More importantly, it is combinatorially faster since it does not have to enumerate lists of indices.

There are obviously numerous ways to create disjoint subdecks from a single deck. One simple method is to simply create $U$ subdecks, each with all the cards of just that one unique card type. For a standard deck of Hanabi, this would mean splitting the deck into a deck with all the red 1s, another with all the green 1s, another with just the blue 5, etc. Each of these subdecks is a monotype deck, for which we know the number of combinations of $K$ cards.

Let any deck $\bm{\rho}$ have $S$ subdecks $\bm{\rho_{s_\delta}}$ such that $\max_{s}\max_{r}\rho_{s_\delta}[r] = 1$ (i.e. each subdeck is a monotype deck with just one unique card type) and $\bm{\rho} = \sum_{s = 1}^S \bm{\rho_{s_{\delta}}}$ (i.e. the subdecks are disjoint and the original deck is composed entirely of the set of disjoint subdecks). Let subdeck $\bm{\rho_{s_{\delta}}}$ have a distinct card type with frequency $R_s$. The number of combinations of $K$ cards from the original deck is a convolution of the possible combinations of cards from each subdeck.

\begin{equation}
    C(\bm{\rho}, K) =  \left (\Conv_{s = 1}^S C(\bm{\rho_{s_{\delta}}}, \cdot)\right )[K]
\end{equation}

\begin{equation}
    C(\bm{\rho}, K) =  \left (\Conv_{s = 1}^S \Pi_{R_{s}}\right )[K]
\end{equation}

Perhaps this is not the most useful formulation for intuition, but its implementation is extremely fast as it is just the convolution of $S$ rectangular functions. Note that this formula is correct for any type of deck representable by a characteristic frequency $\bm{\rho}$.

\subsection{Conclusion}

We can finally get an answer to our original question, as well as others. How many hands are possible in a standard 4-player game of Hanabi?

\begin{equation}
    C(\bm{\rho_0}, 4) = 18,480
\end{equation}

What about the number of combinations of 20 cards dealt to the 5 players in a full-rainbow game of Hanabi?

\begin{equation}
    C([6, 18, 6]^T, 20) = 823,462,074,396
\end{equation}

How about the combinations of starting tiles in a game of Scrabble?

\begin{equation}
    C([5, 10, 1, 4, 3, 1, 2, 1]^T, 7) = 3,199,645
\end{equation}

\section{Conditional Card Distributions}

\pagebreak

\section{Appendix}

\begin{equation}
    C(\bm{\rho_{\bot}}, K) = \frac{(U_{\bot} + K - 1)!}{(U_{\bot} - 1)!K!} \text{ if } K \leq R_{\bot}
\end{equation}
\begin{equation}
    C(\bm{\rho_{\cdot}}, 1) = \frac{(U_{\bot} + 1 - 1)!}{(U_{\bot} - 1)!1!} = U_{\bot}
\end{equation}

\begin{equation}
    C(\bm{\rho_{\cdot}}, 1) = U
\end{equation}


\begin{equation}
    d(k) = \begin{cases} 
      -\mathds{1} & k=1 \\
      \sum_{k=1}^K \rho[k]* & n>1
   \end{cases}
\end{equation}

\begin{equation}
    f(n-1, \bm{\rho}_1^K) = \begin{cases} 
      \sum_{k=1}^K \rho_k & n=1 \\
      \rho_1*f(n - 1, [\rho_1 - 1, \bm{\rho}_2^K]) + \sum_{k=2}^K \rho_k*f(n - 1, [\bm{\rho}_1^{k-1} + 1, \rho_k - 1, \bm{\rho}_{k+1}^K]) & n>1
   \end{cases}
\end{equation}

\begin{equation}
    f(N, M, \bm{\rho}) = \begin{cases} 
      1 & N=0 \text{ or } N=M \\
      \sum_{k=\max (0, \sum \bm{\rho}_1^{M-1})}^{\min(\rho_0, N)}f(N - \rho_0, M - 1, \bm{\rho}_1^{M-1}) & \text{else}
   \end{cases}
\end{equation}
\section{Backup}



We can use this simple result already to bound the possible combinations of Hanabi hands. Adding cards to a deck, whether unique or repeated, can never decrease the number of possible hands. Also, more card type repetition for the same size deck leads to fewer possible hands, all the way down to one possible hand of $K \leq N$ cards and no possible hands of $K > N$ cards if all the cards are identical.  We can bound $C(\bm{\rho}, K)$ from below by allowing only one of each unique card type and from above by considering all cards to be unique. This works from below because going from the lower bounding deck to the true deck involves adding cards, and therefore never decreases the number of combinations. This works from above because going from the higher bounding deck to the true deck involves increasing card type repetition, and therefore never increases the number of combinations.

\begin{equation}
    \binom{U}{K} \leq C(\bm{\rho}, K) \leq \binom{N}{K} 
\end{equation}

Remember that we can derive all relevant deck information from $\bm{\rho}$.

\begin{equation}
    \binom{\sum_{r=1}^{\lvert \bm{\rho}\rvert} \rho[r]}{K} \leq C(\bm{\rho}, K) \leq \binom{\sum_{r=1}^{\lvert\bm{\rho}\rvert} r \rho[r]}{K} 
\end{equation} 

Note that the bounds are tight if and only if all cards are unique (i.e. $\bm{\rho}=[N]$ and so $U=N$). 


One useful quantity we still hope to compute is the number of possible 4 card hands dealt from the standard Hanabi deck.

\begin{equation}
    \binom{25}{4} \leq C(\bm{\rho_0}, 4) \leq \binom{50}{4} 
\end{equation}

\begin{equation}
    12,650 \leq C(\bm{\rho_0}, 4) \leq 230,300 
\end{equation}

What about the total combinations of cards dealt in a full-rainbow, 5-player game?

\begin{equation}
    \bm{\rho_R} = [6, 18, 6]^T
\end{equation}

\begin{equation}
    30,045,015 \leq C(\bm{\rho_R}, 20) \leq 4,191,844,505,805,495
\end{equation}

Clearly, we could use some tighter bounds or, even better, an analytical solution for $C(\bm{\rho}, K)$ when there are non-unique cards.


In the last section, we solved for $C(\bm{\rho}, K)$ if $\lvert \bm{\rho} \rvert=1$ and used the result to bound the solution for any $\bm{\rho}$. In this section, we will do something very similar. Instead of restricting our decks to no-repetition decks, we will restrict our decks to have card types that are all equally repeated. Let $\bm{\rho_\bot}$ be the card frequency of a deck with $U_\bot$ card types where all cards have the same frequency $R_\bot$.\footnote{We are using the $\bot$ notation because it portrays the impulse function that is the characteristic frequency of a uniform-repetition deck.}

\begin{equation}
    \bm{\rho_{\bot}} = U_{\bot} \bm{\delta_{R_{\bot}}}
\end{equation}

That is, $\bm{\rho_{\bot}}$ has just one non-zero element $U_\bot$ at index $R_{\bot}$.

\begin{equation}
    \rho_{\bot}[r] = U_{\bot} \delta_{R_{\bot}}[r] = \begin{cases} U_{\bot} & r=R_{\bot} \\ 0 & \text{else}
    \end{cases}
\end{equation}

Our goal is to count the number of combinations of $K$ cards from the full deck of $N_{\bot}$ cards. In the special case that $K\leq R_\bot$, we are never restricted by running out of any card type when dealing a hand; it is possible to have a hand entirely composed of a single card type. In this special case, the number of combinations is simply the number of combinations of card types with replacement.

The number of ways to select $k$ items from a set of $n$ items with replacement is described by the following formula in the ``$n$ multichoose $k$'' notation \cite{benjamin_quinn_2003}.

\begin{equation}
    \left (\binom{n}{k}\right) = \binom{n + k - 1}{k} = \frac{(n + k - 1)!}{(n - 1)!k!}
\end{equation}

We therefore have an analytical solution for the number of possible hands when all card types in the deck have the same frequency which is at least the number of cards being selected.

\begin{equation}
    C(\bm{\rho_{\bot}}, K) = \left (\binom{U_{\bot}}{K}\right) \text{ if } K \leq R_{\bot} = \frac{(U_{\bot} + K - 1)!}{(U_{\bot} - 1)!K!} \text{ if } K \leq R_{\bot}
\end{equation}

To avoid the constraint that $K\leq R_{\bot}$, we reformulate the problem in terms of compositions. Selecting $K$ items from $U_{\bot}$ options with replacement up to $R_\bot$ times is the same as trying to place $K$ items in exactly $U_{\bot}$ bins such that each bin has a count in the range $[0, R_{\bot}]$. Each count in a bin corresponds to a card of that type being selected.

Restricted compositions are usually discussed where each bin has a count in the range $[1, R]$. Luckily, the number of compositions of $K$ items into exactly $U$ bins where each bin has a non-negative item count is the same as the number of compositions of $K + U$ items into exactly $U$ bins where each bin has a count in the range $[1, R + 1]$. Let this number be represented by $F(n, k, w)$ to match combinatoric notation where $n = K + U$, $k = U$, and $w = R + 1$ for our uses \cite{abramson}.

\begin{equation}
    F(n, k, w) = \sum_{j = 0}^k(-1)^j \binom{k}{j}\binom{n - jw - 1}{k - 1}
\end{equation}

We now have an analytical solution for the number of compositions of $K$ cards from a deck with uniform frequency $\bm{\rho_{\bot}}$ regardless of the number of cards being selected.

\begin{equation}
    C(\bm{\rho_\bot}, K) = F(K + U_\bot, U_\bot, R_\bot + 1)
\end{equation}

\begin{equation}
    C(\bm{\rho_\bot}, K) = \sum_{u = 0}^{U_\bot}(-1)^u \binom{U_\bot}{u}\binom{K + U_\bot - u(R_\bot + 1) - 1}{U_\bot - 1}
\end{equation}

We can improve our bounds from the last section by using this result. We start by defining an operation that never decreases the number of possible combinations of any number of cards from a deck with frequency $\bm{\rho}$. Let $r_<$ be a card type frequencies. Replacing $u$ card types of frequency $r_<$ with $u$ card types of frequency $\hat{r}$ will never decrease the number of possible combinations. This operation adds more cards to the deck while maintaining the number of unique card types, so it should be obvious that this cannot reduce the possible combinations. In fact, the number of combinations is only not increased by this operation if $K \leq r_<$, so the additional repeated cards are irrelevant.

\begin{equation}
    \rho_{N+}[r] = \Delta_{N+}(\bm{\rho}, \hat{r}, r_<, u)[r] = \begin{cases} 
      \rho[r] - u & r=r_< \\
      \rho[r] + u & r=\hat{r} \\
      \rho[r] & \text{else} 
   \end{cases}
\end{equation}

Recall that $R$ is the highest frequency of any single card type in the deck. The operation $\Delta_{N+}(\bm{\rho}, R, r_<, u)$ adds $R - r_<$ cards to $u$ card types with frequency $r_<$ until they have the same frequency as the most repeated card already in the deck. We can repeatedly apply this operation until all card types to have the same frequency $R$. The final result will be a uniform-repetition deck with at least as many combinations for any $K$ as the original deck.

\begin{equation}
    \rho_{\bot_+}[r] = \Delta_{\bot_+}(\bm{\rho})[r]  = \mathbbm{1}_{R}(r) \left (\sum_{s=1}^{R - 1} \rho[s] +\rho[R] \right )
\end{equation}

This operation has an analogous ones for reducing the total number of combinations. We can reduce possible combinations by removing cards of card types with a high frequency. This maintains the number of unique cards in the deck while reducing the size of the deck. 

\begin{equation}
    \rho_{N-}[r] = \Delta_{N-}(\bm{\rho}, \hat{r}, r_>, u)[r]  = \begin{cases} 
      \rho[r] + u & r=\hat{r} \\
      \rho[r] - u & r=r_> \\
      \rho[r] & \text{else} 
   \end{cases}
\end{equation}

And we can again apply this operation repeatedly until we have a uniform-repetition deck with at most the same number of possible combinations. Let $r_{\min}$ be the first non-zero element of $\bm{\rho}$ (i.e. the frequency of the card type with the fewest cards).

\begin{equation}
    \rho_{\bot_-}[r] = \Delta_{\bot_-}(\bm{\rho})[r]  = \mathbbm{1}_{r_{\min}}(r) \left (\rho[r_{\min}] + \sum_{s=r_{\min}+1}^{R} \rho[s] \right )
\end{equation}

We now have new bounds on our desired quantity of combinations.

\begin{equation}
    C(\bm{\Delta_{\bot_-}}(\bm{\rho}), K) \leq C(\bm{\rho}, K) \leq C(\bm{\Delta_{\bot_+}}(\bm{\rho}), K)
\end{equation}

\begin{equation}
    F(K + U_{\bot_-}, U_{\bot_-}, R_{\bot_-} + 1) \leq C(\bm{\rho}, K) \leq F(K + U_{\bot_+}, U_{\bot_+}, R_{\bot_+} + 1)
\end{equation}

Note that our operations don't change the number of unique card types, so $U_{\bot_-} = U = U_{\bot_+}$. Also, $R_{\bot_-} = r_{\min}$ and $R_{\bot_+} = R$.

\begin{equation}
    F(K + U, U, r_{\min} + 1) \leq C(\bm{\rho}, K) \leq F(K + U, U, R + 1)
\end{equation}

We can tighten our bounds on those previous values of interest.

\begin{equation}
    12,650 \leq C(\bm{\rho_0}, 4) \leq 20,450
\end{equation}

\begin{equation}
    30,045,015 \leq C(\bm{\rho_R}, 20) \leq 12,159,022,256,370
\end{equation}

Yes, those pesky lower bounds haven't improved at all since the previous section, but they are actually tighter using this method for any distribution of card frequencies that have $r_{\min}>1$. We can do better still, though!



Let $\bm{\rho}$ be the card type frequency of any deck, and let $\bm{\rho_1}$ and $\bm{\rho_2}$ be disjoint subsets of this deck such that $\bm{\rho} = \bm{\rho_1} + \bm{\rho_2}$. $C(\bm{\rho}, K)$, $C(\bm{\rho_1}, K_1)$ and $C(\bm{\rho_2}, K_2)$ are the number of combinations of $K$, $K_1$, and $K_2$ cards from each of these decks, respectively. Since our full deck is composed entirely of cards from either subdeck, $K = K_1 + K_2$. We are interested in solving for $C(\bm{\rho}, K)$ in terms of $C(\bm{\rho_1}, K_1)$ and $C(\bm{\rho_2}, K_2)$. 

Let $C(\bm{\rho}, K; K_1)$ be the combinations of $K$ cards from the full deck given that there are exactly $K_1$ cards from the first subdeck. There are $C(\bm{\rho_1}, K_1)$ of arranging the $K_1$ cards from the first subdeck, and for each of these arrangements, there are exactly the same $C(\bm{\rho_2}, K - K_1)$ arrangements of $K - K_1$ cards from the second deck. It is important that the two subdecks contain entirely distinct card types, or this would not be true.

\begin{equation}
    C(\bm{\rho}, K; K_1) = C(\bm{\rho_1}, K_1 ; K_1) C(\bm{\rho_2}, K_2; K_1)= C(\bm{\rho_1}, K_1) C(\bm{\rho_2}, K - K_1)
\end{equation}

Of course, for any $K$, $K_1$ can be anything in the range $[0, K]$. These count distinct arrangements, since no hand made of $K_a$ cards from a subdeck can ever be the same as any hand made of $K_b$ cards from the same subdeck if $K_a \ne K_b$. Again, it is important that the two subdecks contain entirely distinct card types, or this would not be true. Therefore, we just need to sum all combinations with $K_1$ cards from the first deck over all values of $K_1$.

\begin{equation}
    C(\bm{\rho}, K) = \sum_{k_1=0}^K C(\bm{\rho}, K; k_1) = \sum_{k_1=0}^K C(\bm{\rho_1}, k_1) C(\bm{\rho_2}, K - k_1)
\end{equation}

This is exactly the convolution of the two combinations functions. 
\begin{equation}
     (f * g)[x] \equiv \sum_{\gamma=-\infty}^{\infty} f[\gamma]g[x - \gamma]
\end{equation}

\begin{equation}
    C(\bm{\rho}, K) = (C(\bm{\rho_1}, \cdot) * C(\bm{\rho_2}, \cdot))[K]
\end{equation}

Since any deck can be repeatedly divided in to two distinct subdecks with no cards of the same type, we can repeatedly apply this method. Let $\bm{\rho_{s}} \in \mathbb{Z}_{+}^R$ be the card type frequency of any set of $S$ disjoint subdecks of $\bm{\rho}$.

\begin{equation}
    \bm{\rho} = \sum_{s = 1}^{S} \bm{\rho_{s}}
\end{equation}

We can then accumulate the number of combinations one subdeck at a time.

\begin{equation}
    C(\bm{\rho}, K) = (C(\bm{\rho_{1}}, K) * (C(\bm{\rho_{2}},K) * ... * (C(\bm{\rho_{S}}, K))...))[K]
\end{equation}

Convolution is linear, and therefore associative.

\begin{equation}
    C(\bm{\rho}, K) = (C(\bm{\rho_{1}}, K) * C(\bm{\rho_{2}},K) * ... * C(\bm{\rho_{S}}, K)) [K]
\end{equation}

\begin{equation}
    C(\bm{\rho}, K) =  (\Conv_{s = 1}^S C(\bm{\rho_{s}}, K))[K]
\end{equation}

A deck with card types in different quantities can be thought of as a group of many decks, each with all card types of the same quantities. Imagine splitting the standard Hanabi deck into three - one with just the 5s, one with the duplicate 2s, 3s, and 4s, and one with the triplicate 3s.

\begin{equation}
    \bm{\rho} = \sum_{r = 1}^{R} \rho_{\bot_r} = \sum_{r = 1}^{R} U_{\bot_r} \bm{\delta_r}
\end{equation}

Since we can calculate the combinations of $K$ cards from cards of uniform-repetition, this decomposition of any deck into many decks with uniform-repetition is very useful. Let's examine what happens when we split any deck into two sub-decks.


\bibliographystyle{plain}
\bibliography{references}
\end{document}
